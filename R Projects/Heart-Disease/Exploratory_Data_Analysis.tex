% Options for packages loaded elsewhere
\PassOptionsToPackage{unicode}{hyperref}
\PassOptionsToPackage{hyphens}{url}
%
\documentclass[
]{article}
\usepackage{amsmath,amssymb}
\usepackage{iftex}
\ifPDFTeX
  \usepackage[T1]{fontenc}
  \usepackage[utf8]{inputenc}
  \usepackage{textcomp} % provide euro and other symbols
\else % if luatex or xetex
  \usepackage{unicode-math} % this also loads fontspec
  \defaultfontfeatures{Scale=MatchLowercase}
  \defaultfontfeatures[\rmfamily]{Ligatures=TeX,Scale=1}
\fi
\usepackage{lmodern}
\ifPDFTeX\else
  % xetex/luatex font selection
\fi
% Use upquote if available, for straight quotes in verbatim environments
\IfFileExists{upquote.sty}{\usepackage{upquote}}{}
\IfFileExists{microtype.sty}{% use microtype if available
  \usepackage[]{microtype}
  \UseMicrotypeSet[protrusion]{basicmath} % disable protrusion for tt fonts
}{}
\makeatletter
\@ifundefined{KOMAClassName}{% if non-KOMA class
  \IfFileExists{parskip.sty}{%
    \usepackage{parskip}
  }{% else
    \setlength{\parindent}{0pt}
    \setlength{\parskip}{6pt plus 2pt minus 1pt}}
}{% if KOMA class
  \KOMAoptions{parskip=half}}
\makeatother
\usepackage{xcolor}
\usepackage[margin=1in]{geometry}
\usepackage{color}
\usepackage{fancyvrb}
\newcommand{\VerbBar}{|}
\newcommand{\VERB}{\Verb[commandchars=\\\{\}]}
\DefineVerbatimEnvironment{Highlighting}{Verbatim}{commandchars=\\\{\}}
% Add ',fontsize=\small' for more characters per line
\usepackage{framed}
\definecolor{shadecolor}{RGB}{248,248,248}
\newenvironment{Shaded}{\begin{snugshade}}{\end{snugshade}}
\newcommand{\AlertTok}[1]{\textcolor[rgb]{0.94,0.16,0.16}{#1}}
\newcommand{\AnnotationTok}[1]{\textcolor[rgb]{0.56,0.35,0.01}{\textbf{\textit{#1}}}}
\newcommand{\AttributeTok}[1]{\textcolor[rgb]{0.13,0.29,0.53}{#1}}
\newcommand{\BaseNTok}[1]{\textcolor[rgb]{0.00,0.00,0.81}{#1}}
\newcommand{\BuiltInTok}[1]{#1}
\newcommand{\CharTok}[1]{\textcolor[rgb]{0.31,0.60,0.02}{#1}}
\newcommand{\CommentTok}[1]{\textcolor[rgb]{0.56,0.35,0.01}{\textit{#1}}}
\newcommand{\CommentVarTok}[1]{\textcolor[rgb]{0.56,0.35,0.01}{\textbf{\textit{#1}}}}
\newcommand{\ConstantTok}[1]{\textcolor[rgb]{0.56,0.35,0.01}{#1}}
\newcommand{\ControlFlowTok}[1]{\textcolor[rgb]{0.13,0.29,0.53}{\textbf{#1}}}
\newcommand{\DataTypeTok}[1]{\textcolor[rgb]{0.13,0.29,0.53}{#1}}
\newcommand{\DecValTok}[1]{\textcolor[rgb]{0.00,0.00,0.81}{#1}}
\newcommand{\DocumentationTok}[1]{\textcolor[rgb]{0.56,0.35,0.01}{\textbf{\textit{#1}}}}
\newcommand{\ErrorTok}[1]{\textcolor[rgb]{0.64,0.00,0.00}{\textbf{#1}}}
\newcommand{\ExtensionTok}[1]{#1}
\newcommand{\FloatTok}[1]{\textcolor[rgb]{0.00,0.00,0.81}{#1}}
\newcommand{\FunctionTok}[1]{\textcolor[rgb]{0.13,0.29,0.53}{\textbf{#1}}}
\newcommand{\ImportTok}[1]{#1}
\newcommand{\InformationTok}[1]{\textcolor[rgb]{0.56,0.35,0.01}{\textbf{\textit{#1}}}}
\newcommand{\KeywordTok}[1]{\textcolor[rgb]{0.13,0.29,0.53}{\textbf{#1}}}
\newcommand{\NormalTok}[1]{#1}
\newcommand{\OperatorTok}[1]{\textcolor[rgb]{0.81,0.36,0.00}{\textbf{#1}}}
\newcommand{\OtherTok}[1]{\textcolor[rgb]{0.56,0.35,0.01}{#1}}
\newcommand{\PreprocessorTok}[1]{\textcolor[rgb]{0.56,0.35,0.01}{\textit{#1}}}
\newcommand{\RegionMarkerTok}[1]{#1}
\newcommand{\SpecialCharTok}[1]{\textcolor[rgb]{0.81,0.36,0.00}{\textbf{#1}}}
\newcommand{\SpecialStringTok}[1]{\textcolor[rgb]{0.31,0.60,0.02}{#1}}
\newcommand{\StringTok}[1]{\textcolor[rgb]{0.31,0.60,0.02}{#1}}
\newcommand{\VariableTok}[1]{\textcolor[rgb]{0.00,0.00,0.00}{#1}}
\newcommand{\VerbatimStringTok}[1]{\textcolor[rgb]{0.31,0.60,0.02}{#1}}
\newcommand{\WarningTok}[1]{\textcolor[rgb]{0.56,0.35,0.01}{\textbf{\textit{#1}}}}
\usepackage{longtable,booktabs,array}
\usepackage{calc} % for calculating minipage widths
% Correct order of tables after \paragraph or \subparagraph
\usepackage{etoolbox}
\makeatletter
\patchcmd\longtable{\par}{\if@noskipsec\mbox{}\fi\par}{}{}
\makeatother
% Allow footnotes in longtable head/foot
\IfFileExists{footnotehyper.sty}{\usepackage{footnotehyper}}{\usepackage{footnote}}
\makesavenoteenv{longtable}
\usepackage{graphicx}
\makeatletter
\def\maxwidth{\ifdim\Gin@nat@width>\linewidth\linewidth\else\Gin@nat@width\fi}
\def\maxheight{\ifdim\Gin@nat@height>\textheight\textheight\else\Gin@nat@height\fi}
\makeatother
% Scale images if necessary, so that they will not overflow the page
% margins by default, and it is still possible to overwrite the defaults
% using explicit options in \includegraphics[width, height, ...]{}
\setkeys{Gin}{width=\maxwidth,height=\maxheight,keepaspectratio}
% Set default figure placement to htbp
\makeatletter
\def\fps@figure{htbp}
\makeatother
\setlength{\emergencystretch}{3em} % prevent overfull lines
\providecommand{\tightlist}{%
  \setlength{\itemsep}{0pt}\setlength{\parskip}{0pt}}
\setcounter{secnumdepth}{-\maxdimen} % remove section numbering
\ifLuaTeX
  \usepackage{selnolig}  % disable illegal ligatures
\fi
\usepackage{bookmark}
\IfFileExists{xurl.sty}{\usepackage{xurl}}{} % add URL line breaks if available
\urlstyle{same}
\hypersetup{
  pdftitle={Unveiling Heart Disease Trends},
  pdfauthor={by Seif Kungulio},
  hidelinks,
  pdfcreator={LaTeX via pandoc}}

\title{Unveiling Heart Disease Trends}
\author{by Seif Kungulio}
\date{}

\begin{document}
\maketitle

\subsection{\texorpdfstring{\textbf{Background and
Introduction}}{Background and Introduction}}\label{background-and-introduction}

Heart disease is a leading global health concern, accounting for a
significant number of deaths each year. It encompasses various
conditions affecting the heart, including coronary artery disease, heart
attacks, and arrhythmias. Early detection and accurate diagnosis are
critical in preventing complications and improving patient outcomes.

The UCI Heart Disease Dataset is a widely used dataset in medical data
analysis, containing patient attributes such as age, cholesterol levels,
blood pressure, chest pain type, and other key indicators. Analyzing
this data can help identify patterns and correlations that contribute to
heart disease.

Data visualization plays a crucial role in understanding complex medical
datasets. By employing exploratory data analysis (EDA) and visualization
techniques, we can uncover hidden trends, compare different risk
factors, and gain insights that support medical decision-making. This
project aims to utilize various data visualization methods to analyze
the dataset and provide a clearer understanding of the factors
influencing heart disease.

By identifying significant risk factors and visualizing their impact,
this project can contribute to early diagnosis strategies, patient
awareness, and potential predictive modeling for heart disease
detection.

\subsection{\texorpdfstring{\textbf{Business
Understanding}}{Business Understanding}}\label{business-understanding}

\subsubsection{\texorpdfstring{\textbf{Problem
Statement}}{Problem Statement}}\label{problem-statement}

This project aims to analyze and visualize the dataset to uncover
trends, correlations, and patterns that distinguish individuals with
heart disease from those without. By leveraging data visualization
techniques, we seek to answer the following questions:

\begin{itemize}
\item
  What are the most influential factors associated with heart disease?
\item
  How do age, cholesterol levels, blood pressure, and other risk factors
  correlate with heart disease presence?
\item
  Can we identify clear patterns in the data that help in early
  diagnosis?
\end{itemize}

The findings from this analysis can serve as a foundation for further
predictive modeling, aiding healthcare professionals in making
data-driven decisions for heart disease risk assessment.

\subsection{\texorpdfstring{\textbf{Data
Understanding}}{Data Understanding}}\label{data-understanding}

The dataset contains various features related to patients' health and
demographic information. We will explore the dataset to understand its
structure and relationships between variables.

\subsubsection{\texorpdfstring{\textbf{1. Data
Dictionary}}{1. Data Dictionary}}\label{data-dictionary}

The dataset contains 14 key attributes that are either numerical or
categorical. These attributes are:

\begin{enumerate}
\def\labelenumi{\arabic{enumi}.}
\tightlist
\item
  \textbf{age:} Age of the patient (numeric)
\item
  \textbf{sex:} Gender of the patient (1 = male, 0 = female)
\item
  \textbf{cp:} Chest pain type (categorical: 1-4)
\item
  \textbf{trestbps:} Resting blood pressure (numeric)
\item
  \textbf{chol:} Serum cholesterol (numeric)
\item
  \textbf{fbs:} Fasting blood sugar (1 = true, 0 = false)
\item
  \textbf{restecg:} Resting electrocardiographic results (categorical)
\item
  \textbf{thalach:} Maximum heart rate achieved (numeric)
\item
  \textbf{exang:} Exercise-induced angina (1 = yes, 0 = no)
\item
  \textbf{oldpeak:} ST depression induced by exercise (numeric)
\item
  \textbf{slope:} The slope of the peak exercise ST segment
  (categorical)
\item
  \textbf{ca:} Number of major vessels (0-3, numeric)
\item
  \textbf{thal:} Thalassemia (categorical: 1 = normal, 2 = fixed defect,
  3 = reversible defect)
\item
  \textbf{target:} Heart disease (1 = disease, 0 = no disease)
\end{enumerate}

\begin{longtable}[]{@{}
  >{\raggedright\arraybackslash}p{(\columnwidth - 6\tabcolsep) * \real{0.2344}}
  >{\raggedright\arraybackslash}p{(\columnwidth - 6\tabcolsep) * \real{0.1562}}
  >{\raggedright\arraybackslash}p{(\columnwidth - 6\tabcolsep) * \real{0.2656}}
  >{\raggedright\arraybackslash}p{(\columnwidth - 6\tabcolsep) * \real{0.3438}}@{}}
\toprule\noalign{}
\begin{minipage}[b]{\linewidth}\raggedright
\textbf{Attribute}
\end{minipage} & \begin{minipage}[b]{\linewidth}\raggedright
\textbf{Type}
\end{minipage} & \begin{minipage}[b]{\linewidth}\raggedright
\textbf{Description}
\end{minipage} & \begin{minipage}[b]{\linewidth}\raggedright
\textbf{Contraints/Rules}
\end{minipage} \\
\midrule\noalign{}
\endhead
\bottomrule\noalign{}
\endlastfoot
\textbf{age} & Numerical & The age of the patient in years & Range: 29 -
77 (Based on the dataset statistics) \\
\textbf{sex} & Categorical & The gender of the patient & Values: 1 =
Male, 0 = Female \\
\textbf{cp} & Categorical & Type of chest pain experienced by the
patient & Values: 1 = Typical angina, 2 = Atypical angina, 3 =
Non-anginal pain, 4 = Asymptomatic \\
\textbf{trestbps} & Numerical & Resting blood pressure of the patient,
measured in mmHg & Range: Typically, between 94 and 200 mmHg \\
\textbf{chol} & Numerical & Serum cholesterol level in mg/dl & Range:
Typically, between 126 and 564 mg/dl \\
\textbf{fbs} & Categorical & Fasting blood sugar level \textgreater{}
120 mg/dl & Values: 1 = True, 0 = False \\
\textbf{restecg} & Categorical & Results of the patient's resting
electrocardiogram & Values: 0 = Normal, 1 = ST-T wave abnormality, 2 =
Probable or definite left ventricular hypertrophy \\
\textbf{thalach} & Numerical & Maximum heart rate achieved during a
stress test & Range: Typically, between 71 and 202 bpm \\
\textbf{exang} & Categorical & Whether the patient experiences
exercise-induced angina & Values: 1 = Yes, 0 = No \\
\textbf{oldpeak} & Numerical & ST depression induced by exercise
relative to rest (an ECG measure) & Range: 0.0 to 6.2 (higher values
indicate more severe abnormalities) \\
\textbf{slope} & Categorical & Slope of the peak exercise ST segment &
Values: 1 = Upsloping, 2 = Flat, 3 = Downsloping \\
\textbf{ca} & Numerical & Number of major vessels colored by fluoroscopy
& Range: 0-3 \\
\textbf{thal} & Categorical & Blood disorder variable related to
thalassemia & Values: 3 = Normal, 6 = Fixed defect, 7 = Reversible
defect \\
\textbf{target} & Categorical & Diagnosis of heart disease & Values: 0 =
No heart disease, 1 = Presence of heart disease \\
\end{longtable}

\begin{center}\rule{0.5\linewidth}{0.5pt}\end{center}

\subsubsection{\texorpdfstring{\textbf{2. Data
Preparation}}{2. Data Preparation}}\label{data-preparation}

\paragraph{\texorpdfstring{\textbf{Data
Loading}}{Data Loading}}\label{data-loading}

Install necessary packages if they are not already installed.

\begin{Shaded}
\begin{Highlighting}[]
\CommentTok{\# For data retrieval from url}
\ControlFlowTok{if}\NormalTok{ (}\SpecialCharTok{!}\FunctionTok{requireNamespace}\NormalTok{(}\StringTok{"RCurl"}\NormalTok{, }\AttributeTok{quietly =} \ConstantTok{TRUE}\NormalTok{)) \{}
  \FunctionTok{install.packages}\NormalTok{(}\StringTok{"RCurl"}\NormalTok{)}
\NormalTok{\}}

\CommentTok{\# For tasks involving data manipulation}
\ControlFlowTok{if}\NormalTok{ (}\SpecialCharTok{!}\FunctionTok{requireNamespace}\NormalTok{(}\StringTok{"dplyr"}\NormalTok{, }\AttributeTok{quietly =} \ConstantTok{TRUE}\NormalTok{)) \{}
  \FunctionTok{install.packages}\NormalTok{(}\StringTok{"dplyr"}\NormalTok{)}
\NormalTok{\}}

\CommentTok{\# For creating visualizations}
\ControlFlowTok{if}\NormalTok{ (}\SpecialCharTok{!}\FunctionTok{requireNamespace}\NormalTok{(}\StringTok{"ggplot2"}\NormalTok{, }\AttributeTok{quietly =} \ConstantTok{TRUE}\NormalTok{)) \{}
  \FunctionTok{install.packages}\NormalTok{(}\StringTok{"ggplot2"}\NormalTok{)}
\NormalTok{\}}

\CommentTok{\# For arranging multiple ggplot2 plots in a single canvas.}
\ControlFlowTok{if}\NormalTok{ (}\SpecialCharTok{!}\FunctionTok{requireNamespace}\NormalTok{(}\StringTok{"patchwork"}\NormalTok{, }\AttributeTok{quietly =} \ConstantTok{TRUE}\NormalTok{)) \{}
  \FunctionTok{install.packages}\NormalTok{(}\StringTok{"patchwork"}\NormalTok{)}
\NormalTok{\}}

\CommentTok{\# for displaying correlation matrices}
\ControlFlowTok{if}\NormalTok{ (}\SpecialCharTok{!}\FunctionTok{requireNamespace}\NormalTok{(}\StringTok{"corrplot"}\NormalTok{, }\AttributeTok{quietly =} \ConstantTok{TRUE}\NormalTok{)) \{}
  \FunctionTok{install.packages}\NormalTok{(}\StringTok{"corrplot"}\NormalTok{)}
\NormalTok{\}}
\end{Highlighting}
\end{Shaded}

Load the required libraries

\begin{Shaded}
\begin{Highlighting}[]
\FunctionTok{library}\NormalTok{(RCurl)}
\FunctionTok{library}\NormalTok{(ggplot2)}
\FunctionTok{library}\NormalTok{(patchwork)}
\FunctionTok{library}\NormalTok{(corrplot)}
\end{Highlighting}
\end{Shaded}

\begin{verbatim}
## corrplot 0.95 loaded
\end{verbatim}

\begin{Shaded}
\begin{Highlighting}[]
\FunctionTok{library}\NormalTok{(dplyr)}
\end{Highlighting}
\end{Shaded}

\begin{verbatim}
## 
## Attaching package: 'dplyr'
\end{verbatim}

\begin{verbatim}
## The following objects are masked from 'package:stats':
## 
##     filter, lag
\end{verbatim}

\begin{verbatim}
## The following objects are masked from 'package:base':
## 
##     intersect, setdiff, setequal, union
\end{verbatim}

\begin{Shaded}
\begin{Highlighting}[]
\FunctionTok{library}\NormalTok{(GGally)}
\end{Highlighting}
\end{Shaded}

\begin{verbatim}
## Registered S3 method overwritten by 'GGally':
##   method from   
##   +.gg   ggplot2
\end{verbatim}

Load the dataset from UCI website using RCurl library

\begin{Shaded}
\begin{Highlighting}[]
\CommentTok{\# Create url object to retrieve the dataset from UCI Machine Learning Repository}
\NormalTok{url }\OtherTok{\textless{}{-}} \StringTok{"https://archive.ics.uci.edu/ml/machine{-}learning{-}databases/heart{-}disease/processed.cleveland.data"}

\CommentTok{\# Read the dataset into a dataframe}
\NormalTok{Heart.df }\OtherTok{\textless{}{-}} \FunctionTok{read.csv}\NormalTok{(}\FunctionTok{url}\NormalTok{(url), }\AttributeTok{header =} \ConstantTok{FALSE}\NormalTok{, }\AttributeTok{na.strings =} \StringTok{"?"}\NormalTok{)}
\end{Highlighting}
\end{Shaded}

Display dimensions of the dataframe

\begin{Shaded}
\begin{Highlighting}[]
\FunctionTok{dim}\NormalTok{(Heart.df)}
\end{Highlighting}
\end{Shaded}

\begin{verbatim}
## [1] 303  14
\end{verbatim}

View the first six rows of the dataset

\begin{Shaded}
\begin{Highlighting}[]
\FunctionTok{head}\NormalTok{(Heart.df)}
\end{Highlighting}
\end{Shaded}

\begin{verbatim}
##   V1 V2 V3  V4  V5 V6 V7  V8 V9 V10 V11 V12 V13 V14
## 1 63  1  1 145 233  1  2 150  0 2.3   3   0   6   0
## 2 67  1  4 160 286  0  2 108  1 1.5   2   3   3   2
## 3 67  1  4 120 229  0  2 129  1 2.6   2   2   7   1
## 4 37  1  3 130 250  0  0 187  0 3.5   3   0   3   0
## 5 41  0  2 130 204  0  2 172  0 1.4   1   0   3   0
## 6 56  1  2 120 236  0  0 178  0 0.8   1   0   3   0
\end{verbatim}

\paragraph{\texorpdfstring{\textbf{Data
Preprocessing}}{Data Preprocessing}}\label{data-preprocessing}

Renaming the column names for clarity

\begin{Shaded}
\begin{Highlighting}[]
\FunctionTok{colnames}\NormalTok{(Heart.df) }\OtherTok{\textless{}{-}} \FunctionTok{c}\NormalTok{(}\StringTok{"age"}\NormalTok{, }\StringTok{"sex"}\NormalTok{, }\StringTok{"cp"}\NormalTok{, }\StringTok{"trestbps"}\NormalTok{, }\StringTok{"chol"}\NormalTok{, }\StringTok{"fbs"}\NormalTok{, }\StringTok{"restecg"}\NormalTok{, }\StringTok{"thalach"}\NormalTok{, }\StringTok{"exang"}\NormalTok{, }\StringTok{"oldpeak"}\NormalTok{, }\StringTok{"slope"}\NormalTok{, }\StringTok{"ca"}\NormalTok{, }\StringTok{"thal"}\NormalTok{, }\StringTok{"target"}\NormalTok{)}
\end{Highlighting}
\end{Shaded}

Display the structure of the dataframe

\begin{Shaded}
\begin{Highlighting}[]
\FunctionTok{str}\NormalTok{(Heart.df)}
\end{Highlighting}
\end{Shaded}

\begin{verbatim}
## 'data.frame':    303 obs. of  14 variables:
##  $ age     : num  63 67 67 37 41 56 62 57 63 53 ...
##  $ sex     : num  1 1 1 1 0 1 0 0 1 1 ...
##  $ cp      : num  1 4 4 3 2 2 4 4 4 4 ...
##  $ trestbps: num  145 160 120 130 130 120 140 120 130 140 ...
##  $ chol    : num  233 286 229 250 204 236 268 354 254 203 ...
##  $ fbs     : num  1 0 0 0 0 0 0 0 0 1 ...
##  $ restecg : num  2 2 2 0 2 0 2 0 2 2 ...
##  $ thalach : num  150 108 129 187 172 178 160 163 147 155 ...
##  $ exang   : num  0 1 1 0 0 0 0 1 0 1 ...
##  $ oldpeak : num  2.3 1.5 2.6 3.5 1.4 0.8 3.6 0.6 1.4 3.1 ...
##  $ slope   : num  3 2 2 3 1 1 3 1 2 3 ...
##  $ ca      : num  0 3 2 0 0 0 2 0 1 0 ...
##  $ thal    : num  6 3 7 3 3 3 3 3 7 7 ...
##  $ target  : int  0 2 1 0 0 0 3 0 2 1 ...
\end{verbatim}

Display the statistical summary of the dataframe

\begin{Shaded}
\begin{Highlighting}[]
\FunctionTok{summary}\NormalTok{(Heart.df)}
\end{Highlighting}
\end{Shaded}

\begin{verbatim}
##       age             sex               cp           trestbps    
##  Min.   :29.00   Min.   :0.0000   Min.   :1.000   Min.   : 94.0  
##  1st Qu.:48.00   1st Qu.:0.0000   1st Qu.:3.000   1st Qu.:120.0  
##  Median :56.00   Median :1.0000   Median :3.000   Median :130.0  
##  Mean   :54.44   Mean   :0.6799   Mean   :3.158   Mean   :131.7  
##  3rd Qu.:61.00   3rd Qu.:1.0000   3rd Qu.:4.000   3rd Qu.:140.0  
##  Max.   :77.00   Max.   :1.0000   Max.   :4.000   Max.   :200.0  
##                                                                  
##       chol            fbs            restecg          thalach     
##  Min.   :126.0   Min.   :0.0000   Min.   :0.0000   Min.   : 71.0  
##  1st Qu.:211.0   1st Qu.:0.0000   1st Qu.:0.0000   1st Qu.:133.5  
##  Median :241.0   Median :0.0000   Median :1.0000   Median :153.0  
##  Mean   :246.7   Mean   :0.1485   Mean   :0.9901   Mean   :149.6  
##  3rd Qu.:275.0   3rd Qu.:0.0000   3rd Qu.:2.0000   3rd Qu.:166.0  
##  Max.   :564.0   Max.   :1.0000   Max.   :2.0000   Max.   :202.0  
##                                                                   
##      exang           oldpeak         slope             ca        
##  Min.   :0.0000   Min.   :0.00   Min.   :1.000   Min.   :0.0000  
##  1st Qu.:0.0000   1st Qu.:0.00   1st Qu.:1.000   1st Qu.:0.0000  
##  Median :0.0000   Median :0.80   Median :2.000   Median :0.0000  
##  Mean   :0.3267   Mean   :1.04   Mean   :1.601   Mean   :0.6722  
##  3rd Qu.:1.0000   3rd Qu.:1.60   3rd Qu.:2.000   3rd Qu.:1.0000  
##  Max.   :1.0000   Max.   :6.20   Max.   :3.000   Max.   :3.0000  
##                                                  NA's   :4       
##       thal           target      
##  Min.   :3.000   Min.   :0.0000  
##  1st Qu.:3.000   1st Qu.:0.0000  
##  Median :3.000   Median :0.0000  
##  Mean   :4.734   Mean   :0.9373  
##  3rd Qu.:7.000   3rd Qu.:2.0000  
##  Max.   :7.000   Max.   :4.0000  
##  NA's   :2
\end{verbatim}

According to the Data Dictionary, the following attributes should be
have binary variables, \texttt{sex}, \texttt{fbs}, \texttt{exang}, and
\texttt{target}. But, some shows to have values besides 0's and 1's.\\
Let's convert binary variables to (0, 1)

\begin{Shaded}
\begin{Highlighting}[]
\NormalTok{Heart.df}\SpecialCharTok{$}\NormalTok{sex }\OtherTok{\textless{}{-}} \FunctionTok{ifelse}\NormalTok{(Heart.df}\SpecialCharTok{$}\NormalTok{sex }\SpecialCharTok{\textgreater{}} \DecValTok{0}\NormalTok{, }\DecValTok{1}\NormalTok{, }\DecValTok{0}\NormalTok{)}
\NormalTok{Heart.df}\SpecialCharTok{$}\NormalTok{fbs }\OtherTok{\textless{}{-}} \FunctionTok{ifelse}\NormalTok{(Heart.df}\SpecialCharTok{$}\NormalTok{fbs }\SpecialCharTok{\textgreater{}} \DecValTok{0}\NormalTok{, }\DecValTok{1}\NormalTok{, }\DecValTok{0}\NormalTok{)}
\NormalTok{Heart.df}\SpecialCharTok{$}\NormalTok{exang }\OtherTok{\textless{}{-}} \FunctionTok{ifelse}\NormalTok{(Heart.df}\SpecialCharTok{$}\NormalTok{exang }\SpecialCharTok{\textgreater{}} \DecValTok{0}\NormalTok{, }\DecValTok{1}\NormalTok{, }\DecValTok{0}\NormalTok{)}
\NormalTok{Heart.df}\SpecialCharTok{$}\NormalTok{target }\OtherTok{\textless{}{-}} \FunctionTok{ifelse}\NormalTok{(Heart.df}\SpecialCharTok{$}\NormalTok{target }\SpecialCharTok{\textgreater{}} \DecValTok{0}\NormalTok{, }\DecValTok{1}\NormalTok{, }\DecValTok{0}\NormalTok{)}
\end{Highlighting}
\end{Shaded}

Check to see if there are missing values in the dataframe

\begin{Shaded}
\begin{Highlighting}[]
\FunctionTok{sapply}\NormalTok{(Heart.df, }\ControlFlowTok{function}\NormalTok{(x) }\FunctionTok{sum}\NormalTok{(}\FunctionTok{is.na}\NormalTok{(x)))}
\end{Highlighting}
\end{Shaded}

\begin{verbatim}
##      age      sex       cp trestbps     chol      fbs  restecg  thalach 
##        0        0        0        0        0        0        0        0 
##    exang  oldpeak    slope       ca     thal   target 
##        0        0        0        4        2        0
\end{verbatim}

From the summary and the table above, there are some missing values in
\texttt{ca} and \texttt{thal} columns.\\
Let's handle the missing values using mean/mode imputation method

\begin{Shaded}
\begin{Highlighting}[]
\CommentTok{\# If missing values exist in \textquotesingle{}ca\textquotesingle{} or \textquotesingle{}thal\textquotesingle{}, handle them using mean/mode imputation}
\NormalTok{Heart.df}\SpecialCharTok{$}\NormalTok{ca[}\FunctionTok{is.na}\NormalTok{(Heart.df}\SpecialCharTok{$}\NormalTok{ca)] }\OtherTok{\textless{}{-}} \FunctionTok{median}\NormalTok{(Heart.df}\SpecialCharTok{$}\NormalTok{ca, }\AttributeTok{na.rm =} \ConstantTok{TRUE}\NormalTok{)}
\NormalTok{Heart.df}\SpecialCharTok{$}\NormalTok{ca[Heart.df}\SpecialCharTok{$}\NormalTok{ca }\SpecialCharTok{==} \StringTok{"?"}\NormalTok{] }\OtherTok{\textless{}{-}} \FunctionTok{median}\NormalTok{(Heart.df}\SpecialCharTok{$}\NormalTok{ca, }\AttributeTok{na.rm =} \ConstantTok{TRUE}\NormalTok{)}
\NormalTok{Heart.df}\SpecialCharTok{$}\NormalTok{thal[}\FunctionTok{is.na}\NormalTok{(Heart.df}\SpecialCharTok{$}\NormalTok{thal)] }\OtherTok{\textless{}{-}} \FunctionTok{median}\NormalTok{(Heart.df}\SpecialCharTok{$}\NormalTok{thal, }\AttributeTok{na.rm =} \ConstantTok{TRUE}\NormalTok{)}
\NormalTok{Heart.df}\SpecialCharTok{$}\NormalTok{thal[Heart.df}\SpecialCharTok{$}\NormalTok{thal }\SpecialCharTok{==} \StringTok{"?"}\NormalTok{] }\OtherTok{\textless{}{-}} \FunctionTok{median}\NormalTok{(Heart.df}\SpecialCharTok{$}\NormalTok{ca, }\AttributeTok{na.rm =} \ConstantTok{TRUE}\NormalTok{)}
\end{Highlighting}
\end{Shaded}

Check for duplicate entries in the dataframe and print them out

\begin{Shaded}
\begin{Highlighting}[]
\NormalTok{dupes }\OtherTok{\textless{}{-}}\NormalTok{ Heart.df[}\FunctionTok{duplicated}\NormalTok{(Heart.df) }\SpecialCharTok{|} \FunctionTok{duplicated}\NormalTok{(Heart.df, }\AttributeTok{fromLast =} \ConstantTok{TRUE}\NormalTok{), ]}
\CommentTok{\# Print or inspect the duplicate entries}
\FunctionTok{print}\NormalTok{(dupes)}
\end{Highlighting}
\end{Shaded}

\begin{verbatim}
##  [1] age      sex      cp       trestbps chol     fbs      restecg  thalach 
##  [9] exang    oldpeak  slope    ca       thal     target  
## <0 rows> (or 0-length row.names)
\end{verbatim}

Convert categorical attributes to factors

\begin{Shaded}
\begin{Highlighting}[]
\CommentTok{\# Define a list of categorical columns with their levels and labels}
\NormalTok{categorical\_columns }\OtherTok{\textless{}{-}} \FunctionTok{list}\NormalTok{(}
  \AttributeTok{sex =} \FunctionTok{list}\NormalTok{(}\AttributeTok{levels =} \FunctionTok{c}\NormalTok{(}\DecValTok{0}\NormalTok{, }\DecValTok{1}\NormalTok{), }\AttributeTok{labels =} \FunctionTok{c}\NormalTok{(}\StringTok{"Female"}\NormalTok{, }\StringTok{"Male"}\NormalTok{)),}
  \AttributeTok{cp =} \FunctionTok{list}\NormalTok{(}\AttributeTok{levels =} \FunctionTok{c}\NormalTok{(}\DecValTok{1}\NormalTok{, }\DecValTok{2}\NormalTok{, }\DecValTok{3}\NormalTok{, }\DecValTok{4}\NormalTok{), }\AttributeTok{labels =} \FunctionTok{c}\NormalTok{(}\StringTok{"Typical Angina"}\NormalTok{, }\StringTok{"Atypical Angina"}\NormalTok{, }\StringTok{"Non{-}Angina"}\NormalTok{, }\StringTok{"Asymptomatic"}\NormalTok{)),}
  \AttributeTok{fbs =} \FunctionTok{list}\NormalTok{(}\AttributeTok{levels =} \FunctionTok{c}\NormalTok{(}\DecValTok{0}\NormalTok{, }\DecValTok{1}\NormalTok{), }\AttributeTok{labels =} \FunctionTok{c}\NormalTok{(}\StringTok{"False"}\NormalTok{, }\StringTok{"True"}\NormalTok{)),}
  \AttributeTok{restecg =} \FunctionTok{list}\NormalTok{(}\AttributeTok{levels =} \FunctionTok{c}\NormalTok{(}\DecValTok{0}\NormalTok{, }\DecValTok{1}\NormalTok{, }\DecValTok{2}\NormalTok{), }\AttributeTok{labels =} \FunctionTok{c}\NormalTok{(}\StringTok{"Normal"}\NormalTok{, }\StringTok{"Wave{-}abnormality"}\NormalTok{, }\StringTok{"Probable"}\NormalTok{)),}
  \AttributeTok{exang =} \FunctionTok{list}\NormalTok{(}\AttributeTok{levels =} \FunctionTok{c}\NormalTok{(}\DecValTok{0}\NormalTok{, }\DecValTok{1}\NormalTok{), }\AttributeTok{labels =} \FunctionTok{c}\NormalTok{(}\StringTok{"No"}\NormalTok{, }\StringTok{"Yes"}\NormalTok{)),}
  \AttributeTok{slope =} \FunctionTok{list}\NormalTok{(}\AttributeTok{levels =} \FunctionTok{c}\NormalTok{(}\DecValTok{1}\NormalTok{, }\DecValTok{2}\NormalTok{, }\DecValTok{3}\NormalTok{), }\AttributeTok{labels =} \FunctionTok{c}\NormalTok{(}\StringTok{"Upsloping"}\NormalTok{, }\StringTok{"Flat"}\NormalTok{, }\StringTok{"Downsloping"}\NormalTok{)),}
  \AttributeTok{thal =} \FunctionTok{list}\NormalTok{(}\AttributeTok{levels =} \FunctionTok{c}\NormalTok{(}\DecValTok{3}\NormalTok{, }\DecValTok{6}\NormalTok{, }\DecValTok{7}\NormalTok{), }\AttributeTok{labels =} \FunctionTok{c}\NormalTok{(}\StringTok{"Normal"}\NormalTok{, }\StringTok{"Fixed Defect"}\NormalTok{, }\StringTok{"Reversible"}\NormalTok{)),}
  \AttributeTok{target =} \FunctionTok{list}\NormalTok{(}\AttributeTok{levels =} \FunctionTok{c}\NormalTok{(}\DecValTok{0}\NormalTok{, }\DecValTok{1}\NormalTok{), }\AttributeTok{labels =} \FunctionTok{c}\NormalTok{(}\StringTok{"No"}\NormalTok{, }\StringTok{"Yes"}\NormalTok{))}
\NormalTok{)}

\CommentTok{\# Apply the factor transformation using a loop}
\ControlFlowTok{for}\NormalTok{ (col }\ControlFlowTok{in} \FunctionTok{names}\NormalTok{(categorical\_columns)) \{}
\NormalTok{  Heart.df[[col]] }\OtherTok{\textless{}{-}} \FunctionTok{factor}\NormalTok{(Heart.df[[col]], }
                            \AttributeTok{levels =}\NormalTok{ categorical\_columns[[col]]}\SpecialCharTok{$}\NormalTok{levels, }
                            \AttributeTok{labels =}\NormalTok{ categorical\_columns[[col]]}\SpecialCharTok{$}\NormalTok{labels)}
\NormalTok{\}}
\end{Highlighting}
\end{Shaded}

\begin{center}\rule{0.5\linewidth}{0.5pt}\end{center}

\subsubsection{\texorpdfstring{\textbf{3. Exploratory Through
Visualization}}{3. Exploratory Through Visualization}}\label{exploratory-through-visualization}

\paragraph{\texorpdfstring{\textbf{Barplots}}{Barplots}}\label{barplots}

Create a function to plot barplots distribution of the categorical
variables

\begin{Shaded}
\begin{Highlighting}[]
\NormalTok{HeartDiseaseBar }\OtherTok{\textless{}{-}} \ControlFlowTok{function}\NormalTok{(var) \{}
  \FunctionTok{ggplot}\NormalTok{(Heart.df, }\FunctionTok{aes}\NormalTok{(}\AttributeTok{x =}\NormalTok{ .data[[var]], }\AttributeTok{fill =}\NormalTok{ target)) }\SpecialCharTok{+}
    \FunctionTok{geom\_bar}\NormalTok{(}\AttributeTok{position =} \StringTok{"dodge"}\NormalTok{) }\SpecialCharTok{+} \FunctionTok{theme\_test}\NormalTok{() }\SpecialCharTok{+}
    \FunctionTok{scale\_fill\_manual}\NormalTok{(}\AttributeTok{values =} \FunctionTok{c}\NormalTok{(}\StringTok{"No"} \OtherTok{=} \StringTok{"\#006000"}\NormalTok{, }\StringTok{"Yes"} \OtherTok{=} \StringTok{"red"}\NormalTok{)) }\SpecialCharTok{+}
    \FunctionTok{labs}\NormalTok{(}\AttributeTok{title =} \FunctionTok{paste}\NormalTok{(}\StringTok{"Distribution of Heart Disease by"}\NormalTok{, var),}
         \AttributeTok{x =}\NormalTok{ var, }\AttributeTok{fill =} \StringTok{"Heart Disease"}\NormalTok{)}
\NormalTok{\}}
\end{Highlighting}
\end{Shaded}

Create the distribution of heart disease by categorical variables

\begin{Shaded}
\begin{Highlighting}[]
\CommentTok{\# Create the plots}
\NormalTok{g1 }\OtherTok{\textless{}{-}} \FunctionTok{ggplot}\NormalTok{(Heart.df, }\FunctionTok{aes}\NormalTok{(}\AttributeTok{x=}\NormalTok{target, }\AttributeTok{fill=}\NormalTok{target))}\SpecialCharTok{+}
  \FunctionTok{geom\_bar}\NormalTok{() }\SpecialCharTok{+} \FunctionTok{theme\_test}\NormalTok{() }\SpecialCharTok{+}
  \FunctionTok{scale\_fill\_manual}\NormalTok{(}\AttributeTok{values =} \FunctionTok{c}\NormalTok{(}\StringTok{"No"} \OtherTok{=} \StringTok{"\#006000"}\NormalTok{, }\StringTok{"Yes"} \OtherTok{=} \StringTok{"red"}\NormalTok{)) }\SpecialCharTok{+}
  \FunctionTok{ggtitle}\NormalTok{(}\StringTok{"Distribution of Heart Disease"}\NormalTok{) }\SpecialCharTok{+}
  \FunctionTok{labs}\NormalTok{(}\AttributeTok{x =} \StringTok{"Heart Disease"}\NormalTok{, }\AttributeTok{fill =} \StringTok{"Heart Disease"}\NormalTok{)}
\NormalTok{g2 }\OtherTok{\textless{}{-}} \FunctionTok{HeartDiseaseBar}\NormalTok{(}\StringTok{"sex"}\NormalTok{)}
\NormalTok{g3 }\OtherTok{\textless{}{-}} \FunctionTok{HeartDiseaseBar}\NormalTok{(}\StringTok{"cp"}\NormalTok{)}
\NormalTok{g4 }\OtherTok{\textless{}{-}} \FunctionTok{HeartDiseaseBar}\NormalTok{(}\StringTok{"fbs"}\NormalTok{)}
\NormalTok{g5 }\OtherTok{\textless{}{-}} \FunctionTok{HeartDiseaseBar}\NormalTok{(}\StringTok{"restecg"}\NormalTok{)}
\NormalTok{g6 }\OtherTok{\textless{}{-}} \FunctionTok{HeartDiseaseBar}\NormalTok{(}\StringTok{"exang"}\NormalTok{)}
\NormalTok{g7 }\OtherTok{\textless{}{-}} \FunctionTok{HeartDiseaseBar}\NormalTok{(}\StringTok{"slope"}\NormalTok{)}
\NormalTok{g8 }\OtherTok{\textless{}{-}} \FunctionTok{HeartDiseaseBar}\NormalTok{(}\StringTok{"thal"}\NormalTok{)}

\CommentTok{\# Combine plot using patchwork}
\NormalTok{(g1 }\SpecialCharTok{|}\NormalTok{ g2) }\SpecialCharTok{/}
\NormalTok{(g3 }\SpecialCharTok{|}\NormalTok{ g4) }\SpecialCharTok{/}
\NormalTok{(g5 }\SpecialCharTok{|}\NormalTok{ g6) }\SpecialCharTok{/}
\NormalTok{(g7 }\SpecialCharTok{|}\NormalTok{ g8)}
\end{Highlighting}
\end{Shaded}

\includegraphics{Exploratory_Data_Analysis_files/figure-latex/unnamed-chunk-14-1.pdf}

\subparagraph{\texorpdfstring{\textbf{Bar Plots Interpretation and
Analysis}}{Bar Plots Interpretation and Analysis}}\label{bar-plots-interpretation-and-analysis}

\paragraph{\texorpdfstring{\textbf{Histogram
Distributions}}{Histogram Distributions}}\label{histogram-distributions}

Create a function to plot histogram distribution of the numerical
variables

\begin{Shaded}
\begin{Highlighting}[]
\NormalTok{HeartDiseaseHist }\OtherTok{\textless{}{-}} \ControlFlowTok{function}\NormalTok{(var1) \{}
  \FunctionTok{ggplot}\NormalTok{(Heart.df, }\FunctionTok{aes}\NormalTok{(}\AttributeTok{x =}\NormalTok{ .data[[var1]], }\AttributeTok{fill =}\NormalTok{ target)) }\SpecialCharTok{+}
    \FunctionTok{geom\_histogram}\NormalTok{(}\AttributeTok{bins =} \DecValTok{15}\NormalTok{) }\SpecialCharTok{+} \FunctionTok{theme\_test}\NormalTok{() }\SpecialCharTok{+}
    \FunctionTok{scale\_fill\_manual}\NormalTok{(}\AttributeTok{values =} \FunctionTok{c}\NormalTok{(}\StringTok{"No"} \OtherTok{=} \StringTok{"\#006000"}\NormalTok{, }\StringTok{"Yes"} \OtherTok{=} \StringTok{"red"}\NormalTok{)) }\SpecialCharTok{+}
    \FunctionTok{labs}\NormalTok{(}\AttributeTok{title =} \FunctionTok{paste}\NormalTok{(}\StringTok{"Distribution of"}\NormalTok{, var1),}
         \AttributeTok{x =}\NormalTok{ var1, }\AttributeTok{fill =} \StringTok{"Heart Disease"}\NormalTok{)}
\NormalTok{\}}
\end{Highlighting}
\end{Shaded}

Create histogram distributions of continuous variables.

\begin{Shaded}
\begin{Highlighting}[]
\CommentTok{\# Create the plots}
\NormalTok{p1 }\OtherTok{\textless{}{-}} \FunctionTok{HeartDiseaseHist}\NormalTok{(}\StringTok{"age"}\NormalTok{)}
\NormalTok{p2 }\OtherTok{\textless{}{-}} \FunctionTok{HeartDiseaseHist}\NormalTok{(}\StringTok{"trestbps"}\NormalTok{)}
\NormalTok{p3 }\OtherTok{\textless{}{-}} \FunctionTok{HeartDiseaseHist}\NormalTok{(}\StringTok{"chol"}\NormalTok{)}
\NormalTok{p4 }\OtherTok{\textless{}{-}} \FunctionTok{HeartDiseaseHist}\NormalTok{(}\StringTok{"thalach"}\NormalTok{)}
\NormalTok{p5 }\OtherTok{\textless{}{-}} \FunctionTok{HeartDiseaseHist}\NormalTok{(}\StringTok{"oldpeak"}\NormalTok{)}

\CommentTok{\# Combine plot using patchwork}
\NormalTok{(p1) }\SpecialCharTok{/}
\NormalTok{(p2 }\SpecialCharTok{|}\NormalTok{ p3) }\SpecialCharTok{/}
\NormalTok{(p4 }\SpecialCharTok{|}\NormalTok{ p5)}
\end{Highlighting}
\end{Shaded}

\includegraphics{Exploratory_Data_Analysis_files/figure-latex/unnamed-chunk-16-1.pdf}

\subparagraph{\texorpdfstring{\textbf{Histogram Plots Interpretation and
Analysis}}{Histogram Plots Interpretation and Analysis}}\label{histogram-plots-interpretation-and-analysis}

\paragraph{\texorpdfstring{\textbf{Boxplots}}{Boxplots}}\label{boxplots}

Create a function to plot side-by-side boxplots

\begin{Shaded}
\begin{Highlighting}[]
\NormalTok{HeartDiseaseBoxplot }\OtherTok{\textless{}{-}} \ControlFlowTok{function}\NormalTok{(var1, var2) \{}
  \FunctionTok{ggplot}\NormalTok{(Heart.df, }\FunctionTok{aes}\NormalTok{(}\AttributeTok{x =}\NormalTok{ .data[[var1]],}
                       \AttributeTok{y =}\NormalTok{ .data[[var2]],}
                       \AttributeTok{fill =}\NormalTok{ .data[[var1]])) }\SpecialCharTok{+}
    \FunctionTok{geom\_boxplot}\NormalTok{() }\SpecialCharTok{+} \FunctionTok{theme\_test}\NormalTok{() }\SpecialCharTok{+}
    \FunctionTok{scale\_fill\_manual}\NormalTok{(}\AttributeTok{values =} \FunctionTok{c}\NormalTok{(}\StringTok{"No"} \OtherTok{=} \StringTok{"\#006000"}\NormalTok{, }\StringTok{"Yes"} \OtherTok{=} \StringTok{"red"}\NormalTok{)) }\SpecialCharTok{+}
    \FunctionTok{labs}\NormalTok{(}\AttributeTok{title =} \FunctionTok{paste}\NormalTok{(}\StringTok{"Boxplot of"}\NormalTok{, var2, }\StringTok{"by"}\NormalTok{, var1),}
         \AttributeTok{x =}\NormalTok{ var1, }\AttributeTok{y =}\NormalTok{ var2, }\AttributeTok{fill =} \StringTok{"Heart Disease"}\NormalTok{)}
\NormalTok{\}}
\end{Highlighting}
\end{Shaded}

Create boxplots of continuous variables.

\begin{Shaded}
\begin{Highlighting}[]
\CommentTok{\# Create the plots}
\NormalTok{p1 }\OtherTok{\textless{}{-}} \FunctionTok{HeartDiseaseBoxplot}\NormalTok{(}\StringTok{"target"}\NormalTok{, }\StringTok{"age"}\NormalTok{)}
\NormalTok{p2 }\OtherTok{\textless{}{-}} \FunctionTok{HeartDiseaseBoxplot}\NormalTok{(}\StringTok{"target"}\NormalTok{, }\StringTok{"trestbps"}\NormalTok{)}
\NormalTok{p3 }\OtherTok{\textless{}{-}} \FunctionTok{HeartDiseaseBoxplot}\NormalTok{(}\StringTok{"target"}\NormalTok{, }\StringTok{"chol"}\NormalTok{)}
\NormalTok{p4 }\OtherTok{\textless{}{-}} \FunctionTok{HeartDiseaseBoxplot}\NormalTok{(}\StringTok{"target"}\NormalTok{, }\StringTok{"thalach"}\NormalTok{)}
\NormalTok{p5 }\OtherTok{\textless{}{-}} \FunctionTok{HeartDiseaseBoxplot}\NormalTok{(}\StringTok{"target"}\NormalTok{, }\StringTok{"oldpeak"}\NormalTok{)}

\CommentTok{\# Combine plot using patchwork}
\NormalTok{(p1 }\SpecialCharTok{|}\NormalTok{ p2) }\SpecialCharTok{/}
\NormalTok{(p3 }\SpecialCharTok{|}\NormalTok{ p4) }\SpecialCharTok{/}
\NormalTok{(p5)}
\end{Highlighting}
\end{Shaded}

\includegraphics{Exploratory_Data_Analysis_files/figure-latex/unnamed-chunk-18-1.pdf}

\subparagraph{\texorpdfstring{\textbf{Box Plots Interpretation and
Analysis}}{Box Plots Interpretation and Analysis}}\label{box-plots-interpretation-and-analysis}

\paragraph{\texorpdfstring{\textbf{Scaterplots}}{Scaterplots}}\label{scaterplots}

\begin{Shaded}
\begin{Highlighting}[]
\NormalTok{HeartDiseaseScatter }\OtherTok{\textless{}{-}} \ControlFlowTok{function}\NormalTok{(point1, point2)\{}
  \FunctionTok{ggplot}\NormalTok{(Heart.df, }\FunctionTok{aes}\NormalTok{(}\AttributeTok{x =}\NormalTok{ .data[[point1]],}
                       \AttributeTok{y =}\NormalTok{ .data[[point2]],}
                       \AttributeTok{color =}\NormalTok{ target)) }\SpecialCharTok{+}
    \FunctionTok{geom\_point}\NormalTok{() }\SpecialCharTok{+} \FunctionTok{theme\_test}\NormalTok{() }\SpecialCharTok{+}
    \FunctionTok{geom\_smooth}\NormalTok{(}\AttributeTok{method =} \StringTok{"lm"}\NormalTok{, }\AttributeTok{se =} \ConstantTok{FALSE}\NormalTok{, }\AttributeTok{color =} \StringTok{"blue"}\NormalTok{, }\AttributeTok{formula =}\NormalTok{ y }\SpecialCharTok{\textasciitilde{}}\NormalTok{ x) }\SpecialCharTok{+}
    \FunctionTok{labs}\NormalTok{(}\AttributeTok{title =} \FunctionTok{paste}\NormalTok{(}\StringTok{"Scatterplot of"}\NormalTok{, point1, }\StringTok{"by"}\NormalTok{, point2),}
       \AttributeTok{x =}\NormalTok{ point1, }\AttributeTok{y =}\NormalTok{ point2, }\AttributeTok{color =} \StringTok{"Heart Disease"}\NormalTok{)}
\NormalTok{\}}
\end{Highlighting}
\end{Shaded}

\begin{Shaded}
\begin{Highlighting}[]
\CommentTok{\# Create the plots}
\NormalTok{p1 }\OtherTok{\textless{}{-}} \FunctionTok{HeartDiseaseScatter}\NormalTok{(}\StringTok{"age"}\NormalTok{, }\StringTok{"oldpeak"}\NormalTok{)}
\NormalTok{p2 }\OtherTok{\textless{}{-}} \FunctionTok{HeartDiseaseScatter}\NormalTok{(}\StringTok{"age"}\NormalTok{, }\StringTok{"chol"}\NormalTok{)}
\NormalTok{p3 }\OtherTok{\textless{}{-}} \FunctionTok{HeartDiseaseScatter}\NormalTok{(}\StringTok{"age"}\NormalTok{, }\StringTok{"trestbps"}\NormalTok{)}
\NormalTok{p4 }\OtherTok{\textless{}{-}} \FunctionTok{HeartDiseaseScatter}\NormalTok{(}\StringTok{"age"}\NormalTok{, }\StringTok{"thalach"}\NormalTok{)}
\NormalTok{p5 }\OtherTok{\textless{}{-}} \FunctionTok{HeartDiseaseScatter}\NormalTok{(}\StringTok{"chol"}\NormalTok{, }\StringTok{"thalach"}\NormalTok{)}
\NormalTok{p6 }\OtherTok{\textless{}{-}} \FunctionTok{HeartDiseaseScatter}\NormalTok{(}\StringTok{"trestbps"}\NormalTok{, }\StringTok{"chol"}\NormalTok{)}
\NormalTok{p7 }\OtherTok{\textless{}{-}} \FunctionTok{HeartDiseaseScatter}\NormalTok{(}\StringTok{"thalach"}\NormalTok{, }\StringTok{"oldpeak"}\NormalTok{)}

\CommentTok{\# Combine plot using patchwork}
\NormalTok{(p1 }\SpecialCharTok{|}\NormalTok{ p2) }\SpecialCharTok{/}
\NormalTok{(p3 }\SpecialCharTok{|}\NormalTok{ p4) }\SpecialCharTok{/}
\NormalTok{(p5 }\SpecialCharTok{|}\NormalTok{ p6) }\SpecialCharTok{/}
\NormalTok{(p7)}
\end{Highlighting}
\end{Shaded}

\includegraphics{Exploratory_Data_Analysis_files/figure-latex/scatterplot-1.pdf}

\subparagraph{\texorpdfstring{\textbf{Scatter Plots Interpretation and
Analysis}}{Scatter Plots Interpretation and Analysis}}\label{scatter-plots-interpretation-and-analysis}

\paragraph{\texorpdfstring{\textbf{Pair
Plots}}{Pair Plots}}\label{pair-plots}

Pairwise relationship between multiple continuous variables

\begin{Shaded}
\begin{Highlighting}[]
\CommentTok{\# Create a colored pair plot for selected variables}
\FunctionTok{ggpairs}\NormalTok{(Heart.df[, }\FunctionTok{c}\NormalTok{(}\StringTok{"age"}\NormalTok{, }\StringTok{"trestbps"}\NormalTok{, }\StringTok{"chol"}\NormalTok{, }
                     \StringTok{"thalach"}\NormalTok{, }\StringTok{"oldpeak"}\NormalTok{, }\StringTok{"target"}\NormalTok{)], }
        \FunctionTok{aes}\NormalTok{(}\AttributeTok{color =}\NormalTok{ target, }\AttributeTok{fill =}\NormalTok{ target))}
\end{Highlighting}
\end{Shaded}

\begin{verbatim}
## `stat_bin()` using `bins = 30`. Pick better value with `binwidth`.
## `stat_bin()` using `bins = 30`. Pick better value with `binwidth`.
## `stat_bin()` using `bins = 30`. Pick better value with `binwidth`.
## `stat_bin()` using `bins = 30`. Pick better value with `binwidth`.
## `stat_bin()` using `bins = 30`. Pick better value with `binwidth`.
\end{verbatim}

\includegraphics{Exploratory_Data_Analysis_files/figure-latex/unnamed-chunk-20-1.pdf}

\subparagraph{\texorpdfstring{\textbf{Pair Plots Interpretation and
Analysis}}{Pair Plots Interpretation and Analysis}}\label{pair-plots-interpretation-and-analysis}

\paragraph{\texorpdfstring{\textbf{Correlation
Matrix}}{Correlation Matrix}}\label{correlation-matrix}

Correlation matrix for continuous variables

\begin{Shaded}
\begin{Highlighting}[]
\CommentTok{\# Selecting only continuous variables}
\NormalTok{continuous\_vars }\OtherTok{\textless{}{-}} \FunctionTok{c}\NormalTok{(}\StringTok{"age"}\NormalTok{, }\StringTok{"trestbps"}\NormalTok{, }\StringTok{"chol"}\NormalTok{, }\StringTok{"thalach"}\NormalTok{, }\StringTok{"oldpeak"}\NormalTok{)}
\NormalTok{continuous\_data }\OtherTok{\textless{}{-}}\NormalTok{ Heart.df }\SpecialCharTok{\%\textgreater{}\%} \FunctionTok{select}\NormalTok{(}\FunctionTok{all\_of}\NormalTok{(continuous\_vars))}

\CommentTok{\# Calculating correlation matrix}
\NormalTok{correlation\_matrix }\OtherTok{\textless{}{-}} \FunctionTok{cor}\NormalTok{(continuous\_data)}

\CommentTok{\# Plotting the correlation matrix}
\FunctionTok{corrplot}\NormalTok{(correlation\_matrix, }\AttributeTok{method =} \StringTok{"circle"}\NormalTok{,}
         \AttributeTok{type =} \StringTok{"lower"}\NormalTok{, }\AttributeTok{tl.col =} \StringTok{"black"}\NormalTok{)}
\end{Highlighting}
\end{Shaded}

\includegraphics{Exploratory_Data_Analysis_files/figure-latex/unnamed-chunk-21-1.pdf}

\subsection{\texorpdfstring{\textbf{Conclusion}}{Conclusion}}\label{conclusion}

\end{document}
